\chapter{Concept and Design}
\label{chapter:concept}
\section{Approaches to Store Social Records}
\subsection{Store the Social Record}
The most intuitive solution would be to switch from the DHT table to the Blockchain, moving each social record through a transaction.
Unfortunately after few considerations, this approach does not seem to be optimal:
\begin{itemize}
	\item Store the entire social record would occupy a high amount of memory in the distribuited node.
	In particular the limit for the data structure is 44kb, but is best practice to keep the records light.
	\item Check the real identity of the uploader would be impossible, one option could be to allow only a specific wallet address to write into it.
	\item Look up would not be efficient.
\end{itemize}


\subsection{Store the Hash}
This solution is inspired by Blockstack architecture and requires to create a DHT to map the GlobalID to the address of the transaction contained inside the blockchain.

The transaction holds in the data field the hash of the Social Record. In this manner, the blockchain has the role of validation infrastructure rather than storage system. The actual data is stored in the distributed hash table. This solution could be vulnerable in case a malicious node tries to restore an old version of the social record by injecting the wrong address in the DHT table, causing a currupted record.

\subsection{Use a third-party solution}
Blockstack offers an already implemented solution using a layered architecture similar to the one proposed in solution 2. It allows also to choose among different blockchain implementations and storage systems. Nevertheless, the underlying idea is the same as solution 2: use the blockchain as a validation system and use a DHT to map ids (keys/GlobalIDs) with the corresponding values (Social Records).

According to Blockstack, the applications built on this system are server-less and decentralized.
Developers start by building a single-page application in Javascript.
Then, instead of plugging the frontend into a centralized API, they plug into an API run by the user.
Developers install a library called 'blockstack.js' and do not have to worry about running servers, maintaining databases, or building out user management systems [add reference].


\section{How to Store the Social Records}
It is important to understand how to write transactions in the blockchain, in particular in an ethereum contract.

Given the fact that the GSLS is an open source not centrally managed server, a safety threat may arise.
In order to ask to the server to conclude a particular transaction, a password is required (otherwise a malicious user could update the social record on behalf of someone else), but of course in case of a malicious node the password could be sniffed.
In order to solve this problem asymmetric keys solutions can be implemented (such as the mentioned above RSA and ECC).


The contract will therefore be a \textit{key-value} array whith the globalId as key and the as value a json object with the folliwing structure:

\paragraph{In case of we store the hashed Social Record}\
\begin{lstlisting}
	{
		publicKey: String,
		hashedSocialRecord: String,
	}
\end{lstlisting}

With this data structure in order to verify the authenticity of the social record will be enough to hash the DHT social record and check it with the variable stored in the blockchain for that particular user.
In order to update the information is required to encript the following object with the user's private key:

\begin{lstlisting}
	{
		globalId: String,
		hashedSocialRecord: String,
	}
\end{lstlisting}

So the contract is able to open the message, check if the globalId match with the one of the user and update the record.


\paragraph{In case of we store the entire Social Record}\
\begin{lstlisting}
	{
		publicKey: string,
		socialRecord: Object
	}
\end{lstlisting}

The record to store is similar, the only difference is that rather than the hashedSocialRecord string, the contract is storying the socialRecord Object (which imply higher costs).
Unfortunately the process to insert the social record for the first time is more complex.
The main reason of this complexity is due to the fact that the GSLS sistem will not look up for the DHT table anymore (and therefore it will be possible to remove it) and the authenticity of the record must be assured by the solidity contract.

The first writing especially, will be extremely difficult with an object as follow:
\begin{lstlisting}
	{
		globalId: string,
		publicKey: string,
		encriptedData: string,
	}
\end{lstlisting}

EncriptedData is a string that can be decrypted with the public key and that will be declared authentic if the result of the decrpytion will be a JSON object as follow.
\begin{lstlisting}
{
	globalId: string,
	socialRecord: Object,
}
\end{lstlisting}

This approach, even though complicated at first, solve a crucial problem and the biggest barrier of the blockchain technology in general: force all the users to run the blockchain.

In fact, if the users don't want to host the ethereum node and cannot rely on the authenticity of the server, they are impossibilitated to update their record.

It is a common practice to trust the server identity of every service (and when it is not possible to trust the service, it is still possiblo to write in the blockchain), but with this approach the writing/reading problem can be solved even without trusting anyone and without hosting any blockchain.

Unfortunately, so far, an asymmetric library in solidity has not been developed yet.


Event though the implementation of Elliptic Curves or RSA keys would be a useful (although expensive) feature, and the Elliptic curve is already implemented in Ethereum in order to create and verify private and public keys, a solidity implementation is not yet available.

Nevertheless there are several proposals and RFCs both for RSA and Elliptic curves \citep{ECImplementation} \citep{RSAImplementation} and have been taken place suggestions of implementing asymmetric keys and many other features in the EVM2.0.
Because of this lack the described process cannot be pursued.

\begin{notation}
	From here second draft of the possible approaches description
	Needs to decide which one to maintain.
\end{notation}

\section{Dictionary / Nomenclature}
\begin{notation}
	TODO: Make it table
\end{notation}
User- The find entity that owns a social record, concrete it, upload, deliver or retrieve it.

Server- The GSLS system can be converted to an Ethereum node with few more functionalities such as an API request interface listening for communication from the client.

Blockchain - the Blockchain is referred as an entity where to store records with transactions.

Transaction- Is an operation that aims at changing the status of the BC through authorized inputs.

Contract - Operations that needs to be performed in particulars contracts when not specified in the doc contract and solidity contract are considered synonyms.

\section{Use wallet without hosting an Ethereum Node}

The last approach that this paper wants to explain requires the users to create an Ethereum wallet but not to host an Ethereum node and can ensure a complete and working implementation without the DHT.

The \textit{key-value} array this time has the user's wallet address as a key and the social record object as a value; this approach also reduce the complexity of the algorithm that needs to be run in solidity and therefore the costs.

The steps to follow are the same as the user participate in the blockchain with an ethereum node: he creates the transaction with all the data but, instead of uploading it in the network, the transaction is sent to the GSLS node and then finally pushed on the network.

This metodology is called offline transaction because the client can create and sign transactions with his ethereum account without running the node and so: being offline. The only limitation of this solution is that the transaction needs be signed with the private key of the account instead that with the usual combination of username - password created with the account.

This could lead to suspicion over this authentication system because the common sense and the documentation suggests not to share the private key, it is important to notice that the private key is not shared to the server and will never leave the local environment.


\section{Possible Approaches}
\subsection{Keep the DHT and verify the authenticity of the Social Record with the blockchain}
The first approach wants to be cheap and light.It will keep the DHT implementation and aims to integrate the Blockchain in order to detect malicious charges in the SR stared in the DHT itself.

\subsubsection{How it works}
\paragraph{Creation \& Update}

This implementation does not require the client to host an Ethereal node, but needs a wallet per each user.

The user needs to download a piece of software to run (eg. exe, epp).

The software creates the Sonic Private Key and Sonic Public Key ( in case of a new user, otherwise will upload his keys), the user must save the private key in a secure location ( otherwise he will loose the possibility to change his SR).
Then the social record is created /uploaded as always and stored in the DHT.
The SR is then Hashed, resulting in a string of small dimension.
The hash is then uploaded in the blockchain through the server in a Key value array with the Old as Key end the hash as a value.

\paragraph{GET}

When a user wants to retrieve information starting from a particular GlobalId, he has to pull the SR of that particular Global iD from the DHT and the hashes from Blockchain.

If the hashes SR of the Blockchain is the same as the hashing of the retrieved from the DHT, the record is pure.

This approach is valid and cheap since only the Hash of the entire SR is stared.
Unfortunately, the solution can only detect unauthorised changes but not rollback from it.

\subsection{Without DHT and Etherereum Accounts for the Users}
This approach tries to merge the security and immutability of the Blockchain without forcing the User to have a wallet or Ethereum account nevertheless is important to keep in mind that the server that contacts the bc must move a node. In our prototype we can imagine the GSLS as a Server hosting in Ethereum Node and implementing some functionalities explained here bellow.

\subsubsection{How it works}

\paragraph{Creation}

The user downloads the software, generates the Sonic Public and Private Key. Once the social record has been created, it is encrypted with the private Key and the following object is send to the server.

(Global ID, Pub Key, Encrypted (SR))

The server with the ( Server wallet) uploads the transaction to the blockchain. The contract stored in Ethereum will decrypt the encrypted SR and store the Key Value array with the Global Id as key and the object (Public key, SR version 1 ) as Value.

\paragraph{Update}

In order to update SR, the User application needs to retrieve the number of version, encrypt the number of version + 1 and send the following object to the server (that will update it to the blockchain). The Solidity contract will decrypt the version and check if it match the version +1 in the Blockchain , if it match, the SR can be decrypted as well and update the record in the BC.

\paragraph{GET}

Get is pretty trivied, asking for a Global Id, the SR will returned. This approach even if valid (although expension is not visible because of ….)

\subsection{User with Ethereum Wallet}

This approach require the Suers to have an Ethereal wallet but not to run an Ethereal node.
This is optimal because it allows the users to update their credentials even without hybrid solutions that port of the the bc (such as ….) Ideally, it is possible to create and change the record even from a random hosting. Pc or mobile device.

It is enough to provide the wallet file .j son file.

\subsubsection{How it works}
\paragraph{Creation \& Update}

Once the wallet file json loaded on the application its is possible to create the social record, sign it with the private key; creating the transaction normally. This signed transaction is then uploaded to the server and then to the BC wallet file json.

Json file with the private key of the Ethereum wallet account ( and other information).

In a Key Value array as follow Gid-SR

\paragraph{GET}

(same as previous)
\begin{notation}

Needs to be done
\end{notation}

This approach force the users to deliver an Ethereal wallet but it conceptually more correct because it uses the offline transaction from —— link to yellow white paper.
